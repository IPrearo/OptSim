% !TEX TS-program = pdflatex
% !TEX encoding = UTF-8 Unicode

% This is a simple template for a LaTeX document using the "article" class.
% See "book", "report", "letter" for other types of document.

\documentclass[11pt]{article} % use larger type; default would be 10pt

\usepackage[utf8]{inputenc} % set input encoding (not needed with XeLaTeX)
\usepackage{amsmath}

%%% Examples of Article customizations
% These packages are optional, depending whether you want the features they provide.
% See the LaTeX Companion or other references for full information.

%%% PAGE DIMENSIONS
\usepackage{geometry} % to change the page dimensions
\geometry{a4paper} % or letterpaper (US) or a5paper or....
% \geometry{margin=2in} % for example, change the margins to 2 inches all round
% \geometry{landscape} % set up the page for landscape
%   read geometry.pdf for detailed page layout information

\usepackage{graphicx} % support the \includegraphics command and options

% \usepackage[parfill]{parskip} % Activate to begin paragraphs with an empty line rather than an indent

%%% PACKAGES
\usepackage{booktabs} % for much better looking tables
\usepackage{array} % for better arrays (eg matrices) in maths
\usepackage{paralist} % very flexible & customisable lists (eg. enumerate/itemize, etc.)
\usepackage{verbatim} % adds environment for commenting out blocks of text & for better verbatim
\usepackage{subfig} % make it possible to include more than one captioned figure/table in a single float
% These packages are all incorporated in the memoir class to one degree or another...

%%% HEADERS & FOOTERS
\usepackage{fancyhdr} % This should be set AFTER setting up the page geometry
\pagestyle{fancy} % options: empty , plain , fancy
\renewcommand{\headrulewidth}{0pt} % customise the layout...
\lhead{}\chead{}\rhead{}
\lfoot{}\cfoot{\thepage}\rfoot{}

%%% SECTION TITLE APPEARANCE
\usepackage{sectsty}
\allsectionsfont{\sffamily\mdseries\upshape} % (See the fntguide.pdf for font help)
% (This matches ConTeXt defaults)

%%% ToC (table of contents) APPEARANCE
\usepackage[nottoc,notlof,notlot]{tocbibind} % Put the bibliography in the ToC
\usepackage[titles,subfigure]{tocloft} % Alter the style of the Table of Contents
\renewcommand{\cftsecfont}{\rmfamily\mdseries\upshape}
\renewcommand{\cftsecpagefont}{\rmfamily\mdseries\upshape} % No bold!


\newcommand{\diverg}{\vec{\nabla}}
\newcommand{\tildee}{ \vec{\tilde{E}}}
\newcommand{\tildeh}{ \vec{\tilde{H}}}
\newcommand{\epsz}{\varepsilon_0}
\newcommand{\muz}{\mu_0}
\newcommand{\couplingdenom}{ \int_{-\infty}^\infty \int_{-\infty}^\infty \hat{z}\cdot\left(\tildee_p^*\times\tildeh_p+\tildee_p\times\tildeh_p^*\right) dxdy }
\newcommand{\couplingdenomone}{ \int_{-\infty}^\infty \int_{-\infty}^\infty \hat{z}\cdot\left(\tildee_1^*\times\tildeh_1+\tildee_1\times\tildeh_1^*\right) dxdy }
\newcommand{\couplingdenomtwo}{ \int_{-\infty}^\infty \int_{-\infty}^\infty \hat{z}\cdot\left(\tildee_2^*\times\tildeh_2+\tildee_2\times\tildeh_2^*\right) dxdy }

%%% END Article customizations

%%% The "real" document content comes below...

\title{Coupling constants derivations}
\author{Ivan Prearo}
%\date{} % Activate to display a given date or no date (if empty),
         % otherwise the current date is printed 

\begin{document}
\maketitle

\section{Derivações}

Das equações de Maxwell aplicadas em guias de onda dielétricos, temos que

\begin{equation}\label{eq:maxwell_guides}
\left\{
\begin{split}
\diverg \times \vec{E} &= -j\omega\mu_0\vec{H} \\
\diverg\times \vec{H} &= j\omega\epsilon_0N_{(x,y)}^2\vec{E}
\end{split}\right.
\end{equation}

Podemos escrever \ref{eq:maxwell_guides} para dois guias de onda como sendo

\begin{equation}\label{eq:maxwell_pguides}
\left\{
\begin{split}
\diverg \times \tildee_p &= -j\omega\muz\tildeh_p \\
\diverg\times \tildeh_p &= j\omega\epsz N_p^2\tildee_p
\end{split}\right.
\end{equation}

Onde $p=1,2$ distingue os campos eletromagnéticos de cada guia de onda separado. Também omitimos a dependência de $N$ ($\tildee,\tildeh$) em $x,y$ ($x,y,z$) por simplificação, mas este fato deve ser mantido em mente para futuras equações, onde esses termos serão integrados no espaço.

Assumindo que os modos guiados pelos dois guias formam uma base completa da solução, podemos escrever

\begin{equation}\label{eq:field_dependancies}
\left\{
\begin{split}
\tildee &= A_{(z)}\tildee_1 + B_{(z)}\tildee_2 \\
\tildeh &= A_{(z)}\tildeh_1 + B_{(z)}\tildeh_2
\end{split}
\right.
\end{equation}

Usando as equações \ref{eq:maxwell_guides} e \ref{eq:field_dependancies} chegamos em

\begin{equation}\label{eq:curl_substitute}
\left\{
\begin{split}
\diverg\times \left( A\tildee_1 + B\tildee_2 \right) &= -j\omega\muz\tildeh \\
\diverg\times \left( A\tildeh_1 + B\tildeh_2 \right) &= j\omega\epsz N^2\tildee
\end{split}
\right.
\end{equation}

Substituindo a identidade da Eq. \ref{eq:rot_av} no lado esquerdo da Eq. \ref{eq:curl_substitute} e usando a Eq. \ref{eq:field_dependancies} em termos que aparecem durante a derivação matemática, podemos chegar em

\begin{equation}\label{eq:AB_dependancies}
\left\{
\begin{split}
\frac{dA}{dz}\hat{z}\times\tildee_1 &+ \frac{dB}{dz}\hat{z}\times\tildee_2 = 0 \\
\frac{dA}{dz}\hat{z}\times\tildeh_1 &-j\omega\epsz\left(N^2-N_1^2\right)A\tildee_1 + \frac{dB}{dz}\hat{z}\times\tildeh_2 -j\omega\epsz\left(N^2-N_2^2\right)B\tildee_2 = 0
\end{split}\right.
\end{equation}

E estes resultados podem ser utilizados nas seguintes equações integrais:

\begin{equation}\label{eq:integral_equations}
\begin{split}
 \int_{-\infty}^\infty \int_{-\infty}^\infty \left[ \tildee_p^*\cdot\right.&\left(\frac{dA}{dz}\hat{z}\times\tildeh_1 - j\omega\epsz\left(N^2-N_1^2\right)A\tildee_1 \right. \\
+ &\left. \frac{dB}{dz}\hat{z}\times\tildeh_2 -j\omega\epsz\left(N^2-N_2^2\right)B\tildee_2\right) \\
 &\left. - \tildeh_p^*\cdot\left( \frac{dA}{dz}\hat{z}\times\tildee_1 + \frac{dB}{dz}\hat{z}\times\tildee_2\right) \right] dxdy = 0
\end{split}
\end{equation}

Para ambos os guias de onda ($p=1,2$). Definindo $I_p$ como o integrando da equação acima, chegamos que

\begin{equation}\label{eq:I_p}
\begin{split}
I_p = &- \frac{dA}{dz}\hat{z}\cdot\left( \tildee_p^*\times\tildeh_1 + \tildee_1\times\tildeh_p^*\right) \\
& - \frac{dB}{dz}\hat{z}\cdot\left( \tildee_p^*\times\tildeh_2 + \tildee_2\times\tildeh_p^*\right) \\
& - j\omega\epsz A\left(N^2-N_1^2\right)\tildee_p^*\cdot\tildee_1 \\
& - j\omega\epsz B\left(N^2-N_2^2\right)\tildee_p^*\cdot\tildee_2
\end{split}
\end{equation}

que pode ser substituído na Eq. \ref{eq:integral_equations} enquanto dividimos por $ \couplingdenom $. Encontrando que, para $p=1$

\begin{equation}\label{eq:Eq_p1}
\begin{split}
\frac{dA}{dz} &+ \frac{dB}{dz}\frac{  \int_{-\infty}^\infty \int_{-\infty}^\infty \hat{z}\cdot\left[ \tildee_2^*\times\tildeh_2 + \tildee_2\times\tildeh_2^* \right] dxdy }{\couplingdenomone} \\
&+ jA \frac{\omega\epsz \int_{-\infty}^\infty \int_{-\infty}^\infty \left(N^2-N_1^2\right)\tildee_1^*\cdot\tildee_1 dxdy}{\couplingdenomone} \\
&+ jB \frac{\omega\epsz \int_{-\infty}^\infty \int_{-\infty}^\infty \left(N^2-N_2^2\right)\tildee_1^*\cdot\tildee_2 dxdy}{\couplingdenomone}
\end{split}
\end{equation}

E ainda, podemos separar a componente da direção de propagação nos campos eletromagnéticos, tendo

\begin{equation*}
\left\{
\begin{split}
\tildee_p &= \vec{E}_pe^{-i\beta_p z} \\
\tildeh_p &= \vec{H}_pe^{-i\beta_p z}
\end{split}
\right.
\end{equation*}

E simplificar utilizando a identidade da Eq. \ref{eq:cross_dot_identity}, resultando nos termos para $p=1$

\begin{equation*}
\left\{
\begin{split}
\int_{-\infty}^\infty\int_{-\infty}^\infty \hat{z}\cdot\left( \vec{E_1^*}e^{j\beta_1z}\times\vec{H_1}e^{-j\beta_1z} + \vec{E_1}e^{j\beta_1z}\times\right.&\left.\vec{H_1^*}e^{j\beta_1z} \right) dxdy = \\
&\int_{-\infty}^\infty\int_{-\infty}^\infty \hat{z}\cdot\left( \vec{E_1^*}\times\vec{H_1} + \vec{E_1}\times\vec{H_1^*} \right) dxdy \\
\int_{-\infty}^\infty\int_{-\infty}^\infty \hat{z}\cdot\left( \vec{E_1^*}e^{j\beta_1z}\times\vec{H_2}e^{-j\beta_1z} + \vec{E_2}e^{j\beta_1z}\times\right.&\left.\vec{H_1^*}e^{j\beta_1z} \right) dxdy = \\
&e^{-j(\beta_2-\beta_1)}\int_{-\infty}^\infty\int_{-\infty}^\infty \hat{z}\cdot\left( \vec{E_1^*}\times\vec{H_2} + \vec{E_2}\times\vec{H_1^*} \right) dxdy \\
\int_{-\infty}^\infty\int_{-\infty}^\infty \left(N^2-N_1^2\right)e^{j\beta_1z}\vec{E_1^*}\cdot\vec{E_1}e^{-j\beta_1z} dxdy& = \\
&\int_{-\infty}^\infty\int_{-\infty}^\infty \left(N^2-N_1^2\right)\vec{E_1^*}\cdot\vec{E_1} dxdy \\
\int_{-\infty}^\infty\int_{-\infty}^\infty \left(N^2-N_2^2\right)e^{j\beta_1z}\vec{E_1^*}\cdot\vec{E_2}e^{-j\beta_2z} dxdy& = \\
&e^{-j(\beta_2-\beta_1)z}\int_{-\infty}^\infty\int_{-\infty}^\infty \left(N^2-N_2^2\right)\vec{E_1^*}\cdot\vec{E_2} dxdy
\end{split}
\right.
\end{equation*}

Portanto a Eq. \ref{eq:Eq_p1} pode ser escrita como

\begin{equation} \label{eq:z_sep_Eq_p1}
\left\{
\begin{split}
\frac{dA}{dz} + \frac{dB}{dz}e^{-j(\beta_2-\beta_1)z}\frac{\int_{-\infty}^\infty\int_{-\infty}^\infty \hat{z}\cdot\left( \vec{E_1^*}\times\vec{H_2} + \vec{E_2}\times\vec{H_1^*} \right) dxdy}{\int_{-\infty}^\infty\int_{-\infty}^\infty \hat{z}\cdot\left( \vec{E_1^*}\times\vec{H_1} + \vec{E_1}\times\vec{H_1^*} \right) dxdy} &\\
+ jA \frac{\omega\epsz  \int_{-\infty}^\infty\int_{-\infty}^\infty \left(N^2-N_1^2\right)\vec{E_1^*}\cdot\vec{E_1} dxdy}{\int_{-\infty}^\infty\int_{-\infty}^\infty \hat{z}\cdot\left( \vec{E_1^*}\times\vec{H_1} + \vec{E_1}\times\vec{H_1^*} \right) dxdy} &\\
+ jBe^{-j(\beta_2-\beta_1)z} \frac{\omega\epsz  \int_{-\infty}^\infty\int_{-\infty}^\infty \left(N^2-N_2^2\right)\vec{E_1^*}\cdot\vec{E_2} dxdy}{\int_{-\infty}^\infty\int_{-\infty}^\infty \hat{z}\cdot\left( \vec{E_1^*}\times\vec{H_1} + \vec{E_1}\times\vec{H_1^*} \right) dxdy} &= 0
\end{split}
\right.
\end{equation}

Finalmente, definindo os coeficientes de acoplamento

\begin{equation}
\label{eq:coupling_coeff}
\left\{
\begin{split}
c_{pq} &= \frac{\int_{-\infty}^\infty\int_{-\infty}^\infty \hat{z} \cdot \left[ \vec{E_p^*}\times\vec{H_q} + \vec{E_q}\times\vec{H_p^*} \right] dxdy}{\couplingdenom} \\
\chi_{p} &= \frac{\omega\epsz\int_{-\infty}^\infty\int_{-\infty}^\infty \left(N^2-N_p^2\right) \vec{E_p^*}\cdot\vec{E_p} dxdy}{\couplingdenom} \\
\kappa_{pq} &= \frac{\omega\epsz\int_{-\infty}^\infty\int_{-\infty}^\infty \left(N^2-N_q^2\right) \vec{E_p^*}\cdot\vec{E_q} dxdy}{\couplingdenom) dxdy}
\end{split}
\right.
\end{equation}

Temos que a Eq. \ref{eq:z_sep_Eq_p1} simplifica para

\begin{equation}
\frac{dA}{dz} + c_{12}\frac{dB}{dz}e^{-j\Delta\beta z} + j\chi_1A + j\kappa_{12}Be^{-j\Delta\beta z} = 0
\end{equation}

onde $\Delta\beta\equiv\beta_2-\beta_1$. De maneira análoga, a partir da Eq. \ref{eq:integral_equations}, mas com $p=2$ obtemos

\begin{equation}
\frac{dB}{dz} + c_{21}\frac{dA}{dz}e^{-j\Delta\beta z} + j\chi_2B + j\kappa_{21}Ae^{-j\Delta\beta z} = 0
\end{equation}

\section{Identidades}


Reescrita do rotacional de uma função escalar dependente apenas em $z$ multiplicada por uma função vetorial:
\begin{equation}\label{eq:rot_av}
\diverg\times\left(a_{(z)}\vec{v}\right) = a_{(z)}\diverg\times\vec{v} + \diverg a_{(z)}\times\vec{v} = a_{(z)}\diverg\times\vec{v} + \frac{da_{(z)}}{dz}\hat{z}\times\vec{v}
\end{equation}

Produtos vetorial e escalar de vetores do tipo $\vec{A}e^{j\phi_1}$ e $\vec{B}e^{-j\phi_2}$
\begin{equation} \label{eq:cross_dot_identity}
\left\{
\begin{split}
\vec{A}e^{j\phi_1}\times \vec{B}e^{-j\phi_2} &= (\vec{A}\times\vec{B})e^{j(\phi_1-\phi_2)} \\
\vec{A}e^{j\phi_1}\cdot \vec{B}e^{-j\phi_2} &= (\vec{A}\cdot\vec{B})e^{j(\phi_1-\phi_2)}
\end{split}
\right.
\end{equation}

\end{document}
