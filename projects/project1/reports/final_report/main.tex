\documentclass[final,12pt,times,twocolumn,authoryear]{elsarticle}


\usepackage{float}
\usepackage{ragged2e}
\usepackage{amssymb}
\usepackage[english]{babel}
\usepackage{graphicx}
\usepackage{geometry}
\usepackage{hyperref}
\usepackage{booktabs}
\usepackage{amsmath}
\usepackage{caption}
\usepackage{subcaption}
\usepackage{graphicx,color}
\usepackage{amsmath}
%\numberwithin{figure}{section}
%\numberwithin{table}{section}

\journal{FI 254}


\newcommand{\diverg}{\vec{\nabla}}
\newcommand{\tildee}{ \vec{\tilde{E}}}
\newcommand{\tildeh}{ \vec{\tilde{H}}}
\newcommand{\epsz}{\varepsilon_0}
\newcommand{\muz}{\mu_0}
\newcommand{\couplingdenom}{ \int_{-\infty}^\infty \int_{-\infty}^\infty \hat{z}\cdot\left(\tildee_p^*\times\tildeh_p+\tildee_p\times\tildeh_p^*\right) dxdy }
\newcommand{\couplingdenomone}{ \int_{-\infty}^\infty \int_{-\infty}^\infty \hat{z}\cdot\left(\tildee_1^*\times\tildeh_1+\tildee_1\times\tildeh_1^*\right) dxdy }
\newcommand{\couplingdenomtwo}{ \int_{-\infty}^\infty \int_{-\infty}^\infty \hat{z}\cdot\left(\tildee_2^*\times\tildeh_2+\tildee_2\times\tildeh_2^*\right) dxdy }



\begin{document}

\begin{frontmatter}

\title{Teoria de modos acoplados e splitter 50/50}

\author[1]{Eduardo A. V. Souza (RA 250950)}
\author[1]{Ivan Prearo (RA 237215)}

\affiliation[1]{organization={Instituto de Física "Gleb Wataghin"},
            addressline={Universidade Estadual de Campinas},
            city={Campinas}, 
            state={São Paulo},
            country={Brasil}}

\begin{abstract}
    
\end{abstract}

\begin{keyword}
    Guias de onda \sep Teoria de Modos Acoplados \sep Beamsplitters
\end{keyword}


\end{frontmatter}

\section{Introdução}
\label{sec:intro}

\section{Guia de onda dielétrico infinito}
\label{sec:slab}

\subsection{Desenvolvimento teórico}
\label{subsec:slab_theory}

\subsection{Análise numérica}
\label{subsec:slab_num}

\section{Guia de onda dielétrico retangular}
\label{subsec:rectangle}

\subsection{Desenvolvimento teórico}
\label{subsec:rectangle_theory}

\subsection{Guia de onda isolado}
\label{subsec:rectangle_solo}

\subsection{Guias de onda acoplados}
\label{subsec:rectangle_coupled}

\subsubsection{Coeficientes de acoplamento}
\label{subsec:coupling_coeff}

As derivações desta subseção foram baseadas no capítulo 4 de \cite{okamoto2021fundamentals}.

Das equações de Maxwell aplicadas em guias de onda dielétricos, tem-se que

\begin{equation}\label{eq:maxwell_guides}
\left\{
\begin{split}
\diverg \times \tildee &= -j\omega\mu_0\tildeh \\
\diverg\times \tildeh &= j\omega\epsilon_0N_{(x,y)}^2\tildee
\end{split}\right.
\end{equation}

As mesmas equações para dois guias de onda distintos podem ser escritas como

\begin{equation}\label{eq:maxwell_pguides}
\left\{
\begin{split}
\diverg \times \tildee_p &= -j\omega\muz\tildeh_p \\
\diverg\times \tildeh_p &= j\omega\epsz N_p^2\tildee_p
\end{split}\right. ,
\end{equation}

\noindent onde $p=1,2$ distingue os campos eletromagnéticos de cada guia de onda isolado. A dependência de $N$ ($\tildee,\tildeh$) em $x,y$ ($x,y,z$) foi emitida por motivos de simplificação, mas deve-se atentar a este fato para as equações seguintes, onde esses termos serão integrados no espaço.

Assumindo que os modos guiados pelos dois guias formam uma base completa da solução conjunta, podemos escrever o campo total como sendo a superposição dos campos eletromagnéticos isolados

\begin{equation}\label{eq:field_dependancies}
\left\{
\begin{split}
\tildee &= A_{(z)}\tildee_1 + B_{(z)}\tildee_2 \\
\tildeh &= A_{(z)}\tildeh_1 + B_{(z)}\tildeh_2
\end{split}
\right.
\end{equation}

Usando as equações \ref{eq:maxwell_guides} e \ref{eq:field_dependancies} chegamos em

\begin{equation}\label{eq:AB_dependancies}
\left\{
\begin{split}
\frac{dA}{dz}\hat{z}\times\tildee_1 &+ \frac{dB}{dz}\hat{z}\times\tildee_2 = 0 \\
\frac{dA}{dz}\hat{z}\times\tildeh_1 &-j\omega\epsz\left(N^2-N_1^2\right)A\tildee_1 \\
+ \frac{dB}{dz}\hat{z}&\times\tildeh_2 -j\omega\epsz\left(N^2-N_2^2\right)B\tildee_2 = 0
\end{split}\right.
\end{equation}

E estes resultados podem ser utilizados nas seguintes equações integrais:

\begin{equation}\label{eq:integral_equations}
 \int_{-\infty}^\infty \int_{-\infty}^\infty \left[ \tildee_p^*\cdot (\ref{eq:AB_dependancies}a) - \tildeh_p^*\cdot(\ref{eq:AB_dependancies}b) \right] dxdy = 0
\end{equation}

Onde (\ref{eq:AB_dependancies}a) e (\ref{eq:AB_dependancies}b) referem-se ao lado esquerdo da primeira e segunda linha da Eq. \ref{eq:AB_dependancies}, respectivamente. Definindo $I_p$ como o integrando da equação acima, é encontrado

\begin{equation}\label{eq:I_p}
\begin{split}
I_p = &- \frac{dA}{dz}\hat{z}\cdot\left( \tildee_p^*\times\tildeh_1 + \tildee_1\times\tildeh_p^*\right) \\
& - \frac{dB}{dz}\hat{z}\cdot\left( \tildee_p^*\times\tildeh_2 + \tildee_2\times\tildeh_p^*\right) \\
& - j\omega\epsz A\left(N^2-N_1^2\right)\tildee_p^*\cdot\tildee_1 \\
& - j\omega\epsz B\left(N^2-N_2^2\right)\tildee_p^*\cdot\tildee_2
\end{split}
\end{equation}

Ainda é possível separar a componente da direção de propagação nos campos eletromagnéticos, ou seja,

\begin{equation*}
\left\{
\begin{split}
\tildee_p &= \vec{E}_pe^{-i\beta_p z} \\
\tildeh_p &= \vec{H}_pe^{-i\beta_p z}
\end{split}
\right.
\end{equation*}

Portanto, ao substituir a Eq. \ref{eq:I_p} e dividir por $ \couplingdenom $, pode-se escrever a Eq. \ref{eq:integral_equations} para $p=1$ como

\begin{equation} \label{eq:z_sep_Eq_p1}
\left\{
\begin{split}
\frac{dA}{dz} + \frac{dB}{dz}e^{-j(\beta_2-\beta_1)z}\frac{\int_{-\infty}^\infty\int_{-\infty}^\infty \hat{z}\cdot\left( \vec{E_1^*}\times\vec{H_2} + \vec{E_2}\times\vec{H_1^*} \right) dxdy}{\int_{-\infty}^\infty\int_{-\infty}^\infty \hat{z}\cdot\left( \vec{E_1^*}\times\vec{H_1} + \vec{E_1}\times\vec{H_1^*} \right) dxdy} &\\
+ jA \frac{\omega\epsz  \int_{-\infty}^\infty\int_{-\infty}^\infty \left(N^2-N_1^2\right)\vec{E_1^*}\cdot\vec{E_1} dxdy}{\int_{-\infty}^\infty\int_{-\infty}^\infty \hat{z}\cdot\left( \vec{E_1^*}\times\vec{H_1} + \vec{E_1}\times\vec{H_1^*} \right) dxdy} &\\
+ jBe^{-j(\beta_2-\beta_1)z} \frac{\omega\epsz  \int_{-\infty}^\infty\int_{-\infty}^\infty \left(N^2-N_2^2\right)\vec{E_1^*}\cdot\vec{E_2} dxdy}{\int_{-\infty}^\infty\int_{-\infty}^\infty \hat{z}\cdot\left( \vec{E_1^*}\times\vec{H_1} + \vec{E_1}\times\vec{H_1^*} \right) dxdy} &= 0
\end{split}
\right.
\end{equation}

\noindent definindo os coeficientes de acoplamento

\begin{equation}
\label{eq:coupling_coeff}
\left\{
\begin{split}
c_{pq} &\equiv \frac{\int_{-\infty}^\infty\int_{-\infty}^\infty \hat{z} \cdot \left[ \vec{E_p^*}\times\vec{H_q} + \vec{E_q}\times\vec{H_p^*} \right] dxdy}{\couplingdenom} \\
\chi_{p} &\equiv \frac{\omega\epsz\int_{-\infty}^\infty\int_{-\infty}^\infty \left(N^2-N_p^2\right) \vec{E_p^*}\cdot\vec{E_p} dxdy}{\couplingdenom} \\
\kappa_{pq} &\equiv \frac{\omega\epsz\int_{-\infty}^\infty\int_{-\infty}^\infty \left(N^2-N_q^2\right) \vec{E_p^*}\cdot\vec{E_q} dxdy}{\couplingdenom}
\end{split} 
\right.
\end{equation}

Tem-se que a Eq. \ref{eq:z_sep_Eq_p1} é simplificada como

\begin{equation}
\frac{dA}{dz} + c_{12}\frac{dB}{dz}e^{-j\Delta\beta z} + j\chi_1A + j\kappa_{12}Be^{-j\Delta\beta z} = 0
\end{equation}

onde $\Delta\beta\equiv\beta_2-\beta_1$. De maneira análoga, a partir da Eq. \ref{eq:integral_equations}, mas com $p=2$ obetém-se

\begin{equation}
\frac{dB}{dz} + c_{21}\frac{dA}{dz}e^{-j\Delta\beta z} + j\chi_2B + j\kappa_{21}Ae^{-j\Delta\beta z} = 0
\end{equation} 

\section{Conclusão}
\label{sec:conclusion}

\section*{Referências Bibliográficas}

\section*{Apêndice}
\label{sec:appendix}

\end{document}